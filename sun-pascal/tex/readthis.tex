%  ANONYMOUS FTP OF THE SUPERPASCAL SOFTWARE
%              PER BRINCH HANSEN
%  School of Computer and Information Science
% Syracuse University, Syracuse, NY 13244, USA
%               29 October 1993
%     Copyright(c) 1993 Per Brinch Hansen

% LATEX PREAMBLE
\documentstyle[twoside,11pt]{article}
\pagestyle{myheadings}
\setlength{\topmargin}{7mm}
\setlength{\textheight}{200mm}
\setlength{\textwidth}{140mm}
\setlength{\oddsidemargin}{14mm}
\setlength{\evensidemargin}{12mm}
\newcommand{\acknowledgements}
  {\section*{Acknowledgements}
   \addcontentsline{toc}{section}
     {Acknowledgements}
  }
\newcommand{\blank}
  {\mbox{\hspace{1.8em}}}
\newcommand{\blankline}
  {\medskip}
\newcommand{\Copyright}
  {Copyright {\copyright}}
\newcommand{\entry}
  {\bibitem{}}
\newcommand{\example}
  {{\it Example:}}
\newcommand{\examples}
  {{\it Examples:}}
\newcommand{\mytitle}[3]
% [title,month,year]
  {\markboth{PER BRINCH HANSEN}{#1}
   \thispagestyle{empty}
   \begin{center}
     {\Large\bf #1}\\
     % TITLE    
     \blankline
       PER BRINCH HANSEN\\
     \blankline
     {\it
       School of Computer and Information Science  \\
       Syracuse University, Syracuse, NY 13244, USA\\
     }
     \blankline
     #2 #3\\
     %  Month Year
   \end{center}
  }
\newcommand{\Superpascal}
  {\it SuperPascal}
\newenvironment{grammar}
  {\begin{small}}
  {\end{small}}
\newenvironment{myabstract}
  {\begin{rm}
     \noindent{\bf Abstract:}}
  {\end{rm}}
\newenvironment{mybibliography}[1]
% [widestlabel]
  {\begin{small}
    \begin{thebibliography}{#1}
      \addcontentsline{toc}
        {section}{References}}
  {  \end{thebibliography}
   \end{small}}
\newenvironment{mykeywords}
  {\begin{small}
     \noindent{\bf Key Words:}}
  {\end{small}}
\newenvironment{mytabular}[1]
% [columns]
  {\begin{small}
     \begin{center}
       \begin{tabular}{#1}}
  {    \end{tabular}
     \end{center}
   \end{small}}
\newenvironment{program}[1]
% [width]
  {\begin{center}
     \begin{minipage}{#1}}
  {  \end{minipage}
   \end{center}}
% Program Indentation
\newcommand{\PA}
  {\noindent}
\newcommand{\PB}
  {\mbox{\hspace{1em}}}
\newcommand{\PC}
  {\mbox{\hspace{2em}}}
\newcommand{\PD}
  {\mbox{\hspace{3em}}}
\newcommand{\PE}
  {\mbox{\hspace{4em}}}

% DOCUMENT TEXT
\begin{document}

\mytitle{ANONYMOUS FTP OF THE SUPERPASCAL SOFTWARE}
  {November}{1993}

\noindent
These instructions describe the {\Superpascal} software,
define the terms and conditions for its use, and explain how
you obtain the software by anonymous FTP.

\begin{center}
  {\bf DEFINITIONS}
\end{center}

\noindent
The {\it SuperPascal} software (hereafter {\it Software}) is
educational software written by Per Brinch Hansen (hereafter
{\it PBH}). The {\it Software} consists of the {\it Manuals},
{\it Programs}, and {\it Scripts} for the programming
language {\Superpascal} invented by {\it PBH}. The {\it
Software} is stored as text in 11 files (hereafter {\it
Files}).

The {\it Manuals}, written by {\it PBH}, are stored as {\LaTeX}
text in 3 {\it Files}:

\begin{itemize}
  \item
  {\it report.tex:} ``The programming language SuperPascal.''
  \item
  {\it user.tex:} ``The SuperPascal user manual.''
  \item
  {\it notes.tex:} ``The SuperPascal software notes.''
\end{itemize}

The {\it Programs}, written by {\it PBH}, are a {\Superpascal}
compiler and interpreter (hereafter {\it Compiler} and {\it
Interpreter}). The {\it Programs} are written in Pascal for
Sun3 and Sun4 workstations running Unix. The {\it Programs}
are stored as Pascal text in 6 {\it Files}:

\begin{itemize}
  \item
  {\it common.p:} The common declarations used by the
  {\it Compiler} and {\it Interpreter}.
  \item
  {\it scan.p:} The {\it Compiler} procedure that performs
  lexical analysis.
  \item
  {\it parse.p:} The {\it Compiler} procedure that performs
  syntax, scope, and type analysis.
  \item
  {\it assemble.p:} The {\it Compiler} procedure that
  assembles interpreted code.
  \item
  {\it compile.p:} The {\it Compiler} program.
  \item
  {\it interpret.p:} The {\it Interpreter} program.
\end{itemize}

The {\it Scripts} are Unix shell scripts stored as text in 2
{\it Files}:

\begin{itemize}
  \item
  {\it sun3.user:} A shell script for compilation of the
  {\it Programs} on a Sun3 workstation under Unix.
  \item
  {\it sun4.user:} A shell script for compilation of the
  {\it Programs} on a Sun4 workstation under Unix.
\end{itemize}

\begin{center}
  {\bf TERMS AND CONDITIONS}
\end{center}

\begin{it}
\noindent
THE MANUALS ARE COPYRIGHTED BY PBH. THE PROGRAMS ARE IN THE
PUBLIC DOMAIN. YOU CAN OBTAIN THE SOFTWARE BY ANONYMOUS FTP.
THE SOFTWARE IS NOT GUARANTEED FOR A PARTICULAR PURPOSE. PBH
SUPPLIES THE SOFTWARE ``AS IS'' WITHOUT ANY WARRANTIES OR
REPRESENTATIONS AND DOES NOT ACCEPT ANY LIABILITIES WITH
RESPECT TO THE SOFTWARE.
YOU (THE USER) ARE RESPONSIBLE FOR SELECTING THE SOFTWARE, AND
FOR THE USE AND RESULTS OBTAINED FROM THE SOFTWARE. YOUR USE
OF THE SOFTWARE INDICATES YOUR ACCEPTANCE OF THESE TERMS AND
CONDITIONS.
\end{it}

\begin{center}
  {\bf FILE TRANSFER PROCEDURE}
\end{center}

\noindent
To obtain the software, use anonymous FTP from the directory
{\it pbh@top.cis.syr.edu}. If your local machine runs Unix,
follow these steps to copy the files:

\blankline

Create an empty directory on your local machine by typing

\begin{center}
  mkdir clone
\end{center}

Enter the local directory by typing

\begin{center}
  cd clone
\end{center}

Select the remote machine by typing

\begin{center}
  ftp top.cis.syr.edu
\end{center}

When prompted for your name, type

\begin{center}
  anonymous
\end{center}

When prompted for your password, type your

\begin{center}
  $<$e-mail address$>$
\end{center}

Enter the remote ftp directory by typing

\begin{center}
  cd pbh
\end{center}

Copy a shell archive that contains the {\it Files} by
typing

\begin{center}
  get software.shar
\end{center}

Leave the remote machine by typing

\begin{center}
  bye
\end{center}

Split the archive into {\it Files} by typing

\begin{center}
  sh software.shar
\end{center}

Your local directory should now contain the Files.

\begin{center}
  {\bf HOW TO GET STARTED}
\end{center}

\noindent
Transform the {\LaTeX} files into PostScript files and
print the {\it Manuals}. Read ``The SuperPascal software
notes" which explain how you compile the {\it Programs}
on Sun3 and Sun4 workstations. Compile the {\it Programs}
into two executable files:

\begin{program}{14.1em}
  {\PA}sc{\blank}An executable {\it Compiler}.   \\
  {\PA}sr{\blank}An executable {\it Interpreter}.\\
\end{program}

Then read "The programming language SuperPascal" and "The
SuperPascal user manual."

\end{document}
